\documentclass{article}

  \title{Opening the third eye: Experiences with Psychedelics}
  \author{Isobel Smith}
  \date{20/11/2016}
  
\usepackage{amsmath}
\usepackage{booktabs}
\usepackage{graphicx}
\usepackage{float}

  \begin{document}
\maketitle

\section{Research Question}

Psychoactive, or psychadelic drugs, such as mushrooms and LSD, are believed by many users to enhance conciousness, giving practitioners a spiritual, or even religious, experience (Letcher, 2007. Watts, 1968). Entrenched in indigenous culture, the use and popularity of these drugs is growing globally, despite prohibition, (Letcher, 2007. Rager, 2013). According to Letcher, practitioners experiences' with these drugs are "weighted", meaning that many people encounter similar hallucinations whilst on these drugs.  According to my own research, psychadelic drugs are amongst the most popular recreational drugs across both genders (see figure 1). What is it about the experiences people have on these drugs that make them so popular? Do people have shared or similar experiences? Is there a common goal or experience that these drug users strive for?\\

In order to answer these questions I will datamine the Erowid experience vaults. This database is built on drug users documenting their own personal experiences, and therefore provides valuable insights into the popularity and the appeal of such substances. 

\section{References}
Letcher, A. 2007. Mad Thoughts on Mushrooms: Discourse and Power in the study of psychedelic conciousness. Anthropology of conciousness. Vol 18. Issue 2. pp 74-97. DOI: 10.1525/ac.2007.18.2.74\\

Rager, J. 2003. Peyote and the Psychedelics: 20th Century Perceptions of the Religious Use of Psychoactive Substances. The Dentison Journal of Religion. Vol. XII: No 1. Retrieved from http://ohio5.openrepository.com/ohio5/bitstream/11282/306650/1/e-journal on 20/11/2016.\\

Watts, A. 1968. Psychedelics and Religious Experience. California Law Review. Vol 56. No, 1. pp 74-85. DOI: 10.2307/3479497. Retrieved from http://www.jstor.org/stable/3479497 on 20/11/2016.

\end{document}